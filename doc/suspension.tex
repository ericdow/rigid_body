\documentclass[9pt]{article}
\usepackage[left=1in, right=1in, bottom=1in, top=1in]{geometry}
\usepackage{amsmath, amssymb, amsthm}
\usepackage{graphicx}
\usepackage{epstopdf}
\usepackage{subfigure}

\title{PGS Suspension Solver}

\begin{document}
\maketitle

The suspension can be modeled as two rigid bodies, the bar and the wheel. The bar is connected to a socket with a prismatic constraint, and the socket moves with some specified motion $x_s(t)$. The wheel connects to the bar with positional constraint. The bar is attached to a spring at its center of mass, which acts as an external force on the bar.

\begin{center}
\includegraphics[scale=0.5]{diagram.pdf}
\end{center}

Each constraint introduces linear and angular impulses:
\[
\overrightarrow{\lambda}_i = 
\left[
\begin{array}{c}
\lambda_{lin} \\
\lambda_{rot}
\end{array} 
\right]
\]
The location of these impulses depends on the constraint, and will change the form of the Jacobian.

The first constraint we consider is the prismatic constraint of the bar/socket system. We have that 
\[
x_b = x_s(t) \implies u_b = \dot{x}_s(t)
\]
\[
\omega_b = 0
\]

The second constraint is the position constraint for the wheel/bar system. 
\[
x_w - \left(x_b + \frac{L}{2}\sin \theta_b\right) = 0 \implies u_w - u_b - \frac{L}{2}\cos \theta_b \omega_b = 0
\]
\[
y_w - \left(y_b - \frac{L}{2}\cos \theta_b\right) = 0 \implies v_w - v_b + \frac{L}{2}\sin \theta_b \omega_b = 0
\]

The laws of motion are
\[
J v_1 = 0 = J (v_0 + M^{-1} F_{ext} \Delta t) + J M^{-1} J^T \lambda
\]
TODO collision constraint

TODO elastic collision can be handled by not setting to zero

TODO Baumgarte stabilization

\end{document}

